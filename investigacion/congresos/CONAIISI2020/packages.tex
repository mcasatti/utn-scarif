\usepackage[T1]{fontenc}
\usepackage[utf8]{inputenc}
% El comando es-noquoting desactiva el uso de < y > para marcar
% comentarios en español. Es porque da conflictos con tikz (graficos)
\usepackage[spanish,es-noquoting]{babel}
%\usepackage[acronym]{glossaries}
% Uso de posicionamiento absoluto en figuras 
\usepackage{float}
% y cambia el estilo (agrega borde)
%\floatstyle{boxed} 
%\restylefloat{figure}
\usepackage[dvipsnames,svgnames,table,x11names]{xcolor}
\usepackage[hidelinks,draft]{hyperref}
\usepackage{graphicx}
\usepackage{tkz-graph}
\usepackage[colorinlistoftodos,textsize=tiny]{todonotes}
\usepackage{cancel}
%\usepackage{subcaption} % Subcaption rompe algunas cuestiones del formato IEEE
\usepackage{enumitem}
\usepackage{tabularx}
\usepackage{amsmath}
\usepackage{tikz}
\usepackage{pgfplots}
\usetikzlibrary{
	arrows,
	decorations.pathmorphing,
	backgrounds,
	positioning,
	calc,
	fit,
	petri,
	quotes,
	% Este paquete produce errores usado en conjunto con tkz-graph
	% babel,	
	arrows.meta,
	decorations.pathreplacing,
	shapes,
	datavisualization,
	plotmarks
}
%----------------------------------------------------------------------------------------
%	BIBLIOGRAPHY AND INDEX
%----------------------------------------------------------------------------------------
\usepackage[
	%style=alphabetic,
	%style=IEEEtran,
	citestyle=numeric,
	sorting=none,
	sortcites=true,
	autopunct=true,
	babel=hyphen,
	hyperref=true,
	abbreviate=false,
	%backref=true,
	%backend=biber,
	bibencoding=utf8,
	defernumbers=true
]{biblatex}
%\addbibresource{bibliografia.bib} % BibTeX bibliography file
%\bibliography{bibliografia.bib}
%\defbibheading{bibempty}{}
%\usepackage[backend=bibtex,defernumbers=true,sorting=none]{biblatex}
%\DeclareBibliographyCategory{cited}
%\AtEveryCitekey{\addtocategory{cited}{\thefield{entrykey}}}
%\addbibresource{bibliografia.bib}
%\nocite{*}

%\usepackage{caption}
\usepackage[framemethod=tikz]{mdframed}
% FUNCION DECLARADA PARA GRAFICAR UNA CAMPANA DE GAUSS
\pgfmathdeclarefunction{gauss}{2}{%
	\pgfmathparse{1/(#2*sqrt(2*pi))*exp(-((x-#1)^2)/(2*#2^2))}%
}