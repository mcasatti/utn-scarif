%\IEEEPARstart{E}
El presente trabajo forma parte del proyecto de investigación y desarrollo que ha sido homologado por la Secretaría de Investigación, Desarrollo y Posgrado de la \utn, desarrollado en el ámbito del \cids, denominado ``Análisis cienciométrico de la producción en investigación científica y tecnológica en la Red de Ingeniería en Informática/Sistemas de Información de CONFEDI'', \proyecto, en el cual se espera caracterizar la producción en investigación científica y tecnológica por medio de la elaboración de una metodología de análisis cienciométrico a partir de la documentación producida por los investigadores, becarios y centros de investigación de las casas de estudios que componen la red.

Dentro de las actividades del proyecto, está considerado el modelado, diseño y desarrollo de una herramienta de análisis para obtener indicadores, métricas y patrones, en base a la información almacenada en una base de datos cienciométrica, que permitan la visualización simple y efectiva de los datos registrados y que oficie de mecanismo de consulta general para elaborar informes y análisis.

Para ello, en este trabajo se presenta un análisis de los principales criterios a tener en cuenta a la hora de diseñar un almacenamiento cienciométrico y al mismo tiempo establecer un primer conjunto de entidades y atributos a tener en cuenta en dicho diseño, en vistas al futuro desarrollo de la herramienta de análisis mencionada.


%El presente trabajo forma parte del proyecto de investigación y desarrollo que ha sido homologado por la Secretaría de Investigación, Desarrollo y Posgrado de la \utn, desarrollado en el ámbito del \cids, denominado ``Análisis cienciométrico de la producción en investigación científica y tecnológica en la Red de Ingeniería en Informática/Sistemas de Información de CONFEDI'', (SIUTNCO0007848).
