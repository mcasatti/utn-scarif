Los sistemas de análisis cienciométrico prometen un amplio abanico de posibilidades, tanto para los autores como para las instituciones dedicadas a la investigación y desarrollo y son herramientas valiosas al momento de establecer políticas de ciencia y tecnología con base fáctica.

Se espera que los criterios identificados en este trabajo, permitan el modelado y el diseño de una base de datos que caracterice la información necesaria para poder desarrollar una herramienta de análisis que posibilite obtener indicadores, métricas y patrones relacionadas a la producción en investigación científica y tecnológica. 

Se prevé, en un futuro cercano, procesar los formatos documentales producidos durante las diversas ediciones del congreso CoNaIISI para, posteriormente, analizar los formatos de artículos publicados en otros medios y congresos, para validar los lineamientos generales vertidos en este artículo y ampliar o corregir el conjunto de entidades y atributos.

Así mismo, se comenzará la implementación física de la base de datos, luego de realizar un análisis de las bases de datos de grafos disponibles, sus características y prestaciones.

Actualmente se encuentran en desarrollo algoritmos de extracción de información directamente de archivos PDF, debido a que éste es el formato por excelencia en el cual se almacenan artículos científicos en repositorios institucionales, actas de congresos y publicaciones periódicas de diversos tipos. Dichos algoritmos se adecuarán, posteriormente, a otros formatos de documento a medida que se vayan estudiando sus características.

Una vez desarrollados los algoritmos básicos para procesar los documentos de CoNaIISI se procederá a importar una muestra representativa al almacenamiento de grafos, para determinar la validez y robustez del diseño propuesto.
