% paper=a4paper (default),letterpaper,a5paper,b5paper,executivepaper,legalpaper
% font=
% fontsize=10pt, 11pt, 12pt (10pt por defecto)
% documentclass=article,report,book,letter,"ieeetran"
% draft=draft (no carga imagenes, pero indica lugares)
% columns=onecolumn(default), twocolumn
% margenes=oneside,twoside
\documentclass[
    a4paper,
    12pt
]{article}

\usepackage{times}
\usepackage{amsmath}
%\usepackage[options]{paquete}
%\usepackage[spanish]{babel}
%\usepackage[utf8]{inputenc}
%\usepackage[T1]{fontenc}
%\usepackage{graphicx}
%\usepackage{amsmath}

\title{Titulo del trabajo}
\date{\today}
\author{Autor o autores}

\usepackage{lipsum}

\begin{document}
	\maketitle
	\begin{abstract}
		\lipsum[1-2]
	\end{abstract}
	%\clearpage

    \section{Formulas inline}
    
    Esta fórmula está embebida en un párrafo normal $f(x) = x^2$ pero con sintaxis de matemática.
    
    Ahora una fórmula inline más compleja: $f(x) = \ln \frac{x^2}{\sqrt{\pi}}$

    \section{Ecuaciones}
    
    \begin{equation}
        f(x) = x^2
    \end{equation}
    \begin{equation}
        g(x) = \frac{1}{x}\\
    \end{equation}
    \begin{equation}
        F(x) = \int^a_b \frac{1}{3}x^3
    \end{equation}

    \section{Matrices}
    
    $
    \begin{matrix}
        1 & 0\\
        0 & 1
    \end{matrix}
    $
    
    $
    \left[
    \begin{matrix}
        1 & 0\\
        0 & 1
    \end{matrix}
    \right]
    $
	%\includegraphics[scale=0.1]{ship1}
	%\includegraphics[width=250px]{tech1}
	
\end{document}