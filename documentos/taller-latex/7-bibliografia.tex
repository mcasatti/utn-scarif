% paper=a4paper (default),letterpaper,a5paper,b5paper,executivepaper,legalpaper
% font=
% fontsize=10pt, 11pt, 12pt (10pt por defecto)
% documentclass=article,report,book,letter,"ieeetran"
% draft=draft (no carga imagenes, pero indica lugares)
% columns=onecolumn(default), twocolumn
% margenes=oneside,twoside
\documentclass[
    a4paper,
    12pt
]{article}

\usepackage{times}
%\usepackage[options]{paquete}
%\usepackage[spanish]{babel}
%\usepackage[utf8]{inputenc}
%\usepackage[T1]{fontenc}
%\usepackage{graphicx}
%\usepackage{amsmath}

\title{Titulo del trabajo}
\date{\today}
\author{Autor o autores}

\usepackage{lipsum}

\usepackage[
backend=biber,
style=ieee,
sorting=none
]{biblatex}
\addbibresource{biblio.bib}

\begin{document}
	\maketitle
%	\begin{abstract}
%		\lipsum[1-2]
%	\end{abstract}
	%\clearpage

    \tableofcontents

    En un texto cualquiera se puede incluir una cita a un autor\cite{ARTICLE:1}, con cualquier tipo de referencia\cite{DUMMY:1} y el lenguaje se encarga de enumerar y ordenar las referencias bibliográficas.
    
    En donde se quiera generar la tabla de referencias bibliograficas se debe indicar el comando
    
    Es posible citar las mismas referencias bibliográficas\cite{ARTICLE:1} en distintos lugares\cite{ARTICLE:1}, simplemente incluyendo el mismo tag de referencia\cite{ARTICLE:1}.
    
    Todas las veces que se cite algo\cite{DUMMY:1} va a tener el mismo número de referencia que va asociado al texto. La misma cita se puede utilizar de diferentes maneras, por ejemplo solamente el autor\autocite{DUMMY:1}.
    
    Un cambio en el estilo utilizado cambia completamente la forma de generar las referencias.
    
    
    \printbibliography

\end{document}